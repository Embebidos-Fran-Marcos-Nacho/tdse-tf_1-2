\documentclass[12pt,a4paper]{article}

\usepackage[utf8]{inputenc}
\usepackage[T1]{fontenc}
\usepackage[spanish]{babel}
\usepackage{lmodern}
\usepackage{geometry}
\geometry{margin=2.5cm}

\usepackage{graphicx}
\usepackage{float}
\usepackage[unicode,hidelinks]{hyperref}
\usepackage{url}
\usepackage{tabularx}
\usepackage{booktabs}
\usepackage{enumitem}
\usepackage{amsmath}

\addto\captionsspanish{\renewcommand{\contentsname}{Índice General}}

\setlist[itemize]{noitemsep, topsep=2pt}
\setlist[enumerate]{noitemsep, topsep=2pt}

\title{Dimmer + Switch (Ventilador \& Luces)\\Control de ventilador y luces de línea (220 V) desde pared y vía Bluetooth}
\author{Ignacio Ezequiel Cavicchioli \and Francisco Javier Moya}
\date{25/01/2026}

\begin{document}
\maketitle
\tableofcontents
\newpage
\section{Dimmer + Switch (Ventilador \& Luces)}
Control de ventilador y luces de línea (220 V) desde pared y vía Bluetooth

\textbackslash{}begin\{center\}
\textbackslash{}begin\{center\}
\textbackslash{}includegraphics[width=0.6\textbackslash{}linewidth]\{https://github.com/Embebidos-Fran-Marcos-Nacho/tdse-tf\textbackslash{}\_1-2/blob/08290a7a62c8a7d3fcd22fc57871dafbbf35ab15/logo-fiuba.png\}
\textbackslash{}end\{center\}
\textbf{UNIVERSIDAD DE BUENOS AIRES\textbackslash{}\}\textbackslash{}\textbackslash{}\textbf{Facultad de Ingeniería\textbackslash{}\}\textbackslash{}\textbackslash{}\textbf{TA134 – Sistemas Embebidos\textbackslash{}\}\textbackslash{}\textbackslash{}Curso 1 – Grupo 2
\textbackslash{}end\{center\}

\subsection{Autores}
\begin{itemize}
\item Ignacio Ezequiel Cavicchioli — Legajo 109428
\item Francisco Javier Moya — Legajo 109899
\end{itemize}

\textbf{Fecha:\} 25/01/2026
\textbf{Cuatrimestre de cursada:\} 2do cuatrimestre 2025

\textit{Trabajo realizado entre diciembre 2025 y febrero 2026.\}

\hrulefill


\subsection{Resumen}

Se desarrolló un sistema embebido para control de luz y ventilador de red (220 VAC), con:
\begin{itemize}
\item Control local por pulsadores y potenciómetro.
\item Telemetría por Bluetooth con módulo HC-06.
\item Sincronización por cruce por cero.
\item Almacenamiento persistente en flash interna del STM32.
\end{itemize}

El hardware se implementó en dos placas (shield de control y placa de potencia/dimmer), evitando protoboard y cableado Dupont para la integración final. La única excepción es el uso de leds en paralelo con los bulbos de luz requeridos en las pruebas de potencia; la tensión no es suficiente como para encenderlos, por lo que se usaron leds en paralelo como indicadores.
El firmware se implementó en una NUCLEO-F103RB con arquitectura modular de tareas y máquina de estados para modos de inicialización, operación normal y falla segura.

Esta memoria documenta los requisitos, el diseño de hardware y firmware, los ensayos realizados y el estado final de cumplimiento.

\hrulefill

\subsection{Registro de versiones}

\begin{table}[h]
\centering
\begin{tabularx}{\linewidth}{clX}
\toprule
Revisión & Cambios realizados & Fecha \\
\midrule
1.0 & Reescritura integral de la memoria, alineada a pautas de entrega final & 17/02/2026 \\
1.1 & Completar con mediciones de consumo, WCET y factor de uso CPU & 17/02/2026 \\
1.2 & Completar con permalinks definitivos de imágenes y link de video & 17/02/2026 \\
1.2 & Entrega N°1 & 17/02/2026 \\
1.3 & Correcciones Entrega N°2 & 19/02/2026 \\
\bottomrule
\end{tabularx}
\end{table}

\hrulefill

\section{Introducción general}

\subsection{Análisis de necesidad y objetivo}

El proyecto busca resolver una necesidad concreta de control de cargas de 220 VAC (luz y ventilador) desde una interfaz de pared, agregando telemetría inalámbrica sin depender de la red Wi-Fi doméstica.

Objetivos principales:
\begin{itemize}
\item Implementar un prototipo funcional y seguro de control de luz/ventilador.
\item Usar una arquitectura modular en STM32.
\item Tener persistencia de estado en la memoria flash.
\end{itemize}

\subsection{Productos comparables}

Se analizaron dos tipos de soluciones comerciales disponibles en la Argentina:

\begin{enumerate}
\item \textbf{Ventilador con control remoto IR/RF\}
Existen en el mercado local ventiladores controlados por control remoto dedicado.
Este tipo de control funciona correctamente, pero presenta limitaciones importantes:
\end{enumerate}
\begin{itemize}
\item Solo tiene control remoto, no tiene control fijo.
\item No ofrece conectividad con el celular.
\item No guarda configuraciones ni estados previos del ventilador.
\item Solo tiene 3 velocidades de ventilador.
\end{itemize}


\begin{enumerate}
\item \textbf{Controladores disponibles internacionalmente (Amazon)\}
En el mercado internacional existen productos más avanzados, capaces de integrar control de luces y ventilador, conectividad Wi-Fi, y aplicaciones móviles.
Sin embargo:
\end{enumerate}
\begin{itemize}
\item Tienen costos significativamente más altos o no cuentan con disponibilidad local inmediata.
\item En general los que usan wi-fi no tienen tecla y representan una amenaza a la seguridad de la red doméstica del usuario.
\end{itemize}

Ante la gran brecha de funcionalidad entre estos dos dispositivos, se optó por utilizar una interfaz local combinada con un módulo Bluetooth clásico HC-06. Esta solución híbrida prioriza la simplicidad de integración, combinando la comodidad del control de pared con la telemetría inalámbrica por medio de bleutooth. La siguiente sección brinda más detalles sobre estas decisiones de diseño.

\subsection{Justificación del enfoque técnico}

Se eligió Bluetooth clásico (HC-06) por:
\begin{itemize}
\item Menor complejidad de despliegue que Wi-Fi.
\item Facilidad de integración con la app realizada en MIT App Inventor.
\item Disponibilidad de herramientas de depuración por UART.
\end{itemize}

Se mantuvo un alcance acotado para cumplir entrega:
\begin{itemize}
\item La app móvil recibe telemetría binaria de 2 bytes.
\item El control principal de actuadores se mantiene en interfaz local.
\end{itemize}

En una futura versión, el producto debería permitir el control por medio de la conexión inalámbrica, equiparandolo a la solución comercial mostrada más completa.

\subsection{Alcance y limitaciones}

Alcance implementado:
\begin{itemize}
\item Encendido/apagado de luz por botones físicos.
\item Ajuste de velocidad del ventilador por potenciómetro.
\item Envío de telemetría por HC-06 (2 bytes).
\item Estado de falla segura y persistencia básica en flash.
\end{itemize}

Fuera de alcance actual:
\begin{itemize}
\item Control remoto completo de actuadores desde app.
\end{itemize}

\hrulefill

\section{Introducción específica}

\subsection{Requisitos (versión final del informe de avances)}

\begin{table}[h]
\centering
\begin{tabularx}{\linewidth}{llX}
\toprule
Grupo & ID & Descripción \\
\midrule
Control & 1.1 & El sistema permitirá encender y apagar las luces mediante un botón físico. \\
 & 1.2 & El sistema permitirá ajustar la velocidad del ventilador mediante un potenciómetro. \\
 & 1.3 & El sistema permitirá ver el estado del ventilador y las luces vía Bluetooth. \\
Bluetooth & 2.1 & El sistema contará con un DIP switch para habilitar o deshabilitar el Bluetooth. \\
 & 2.2 & El DIP switch permitirá seleccionar configuraciones o canales del módulo Bluetooth. \\
Indicadores & 3.1 & El sistema contará con LEDs que indiquen el estado del Bluetooth. \\
 & 3.2 & El sistema contará con un buzzer para señalizar eventos del sistema. \\
Memoria & 4.1 & El sistema deberá guardar en memoria flash el último valor de PWM utilizado. \\
 & 4.2 & El sistema deberá restaurar automáticamente el último valor guardado al encender. \\
Seguridad eléctrica & 5.1 & El sistema deberá operar de forma segura sobre cargas de 220 VAC. \\
Aplicación móvil & 6.1 & La aplicación dará información sobre los estados disponibles, que incluyen la velocidad del ventilador y el estado de luces. \\
 & 6.2 & El sistema deberá evitar conflictos entre el control físico y la comunicación Bluetooth, incluyendo conflictos de timings. \\
\bottomrule
\end{tabularx}
\end{table}

\subsection{Casos de uso}

\subsubsection{Caso de uso 1: Control local de luz}

\begin{table}[h]
\centering
\begin{tabular}{ll}
\toprule
Elemento & Definición \\
\midrule
Disparador & Pulsación de botón ON (\texttt{PC12\}) o OFF (\texttt{PC9\}). \\
Precondiciones & Sistema en modo normal, hardware operativo. \\
Flujo básico & Debounce de botón -> evento -> actualización de estado de luz -> actualización de salida TRIAC -> solicitud de guardado en flash -> telemetría BT de cambio. \\
Alternativas & Si falla persistencia y modo estricto activo: transición a \texttt{FAULT\}. \\
\bottomrule
\end{tabular}
\end{table}

\subsubsection{Caso de uso 2: Ajuste local de ventilador}

\begin{table}[h]
\centering
\begin{tabular}{ll}
\toprule
Elemento & Definición \\
\midrule
Disparador & Cambio en potenciómetro (\texttt{PA0\}). \\
Precondiciones & ADC operativo, sistema en modo normal. \\
Flujo básico & Muestreo ADC -> mapeo a porcentaje -> cálculo de \texttt{fan\textbackslash{}\_delay\textbackslash{}\_us\} -> actualización de temporización de disparo TRIAC. \\
Alternativas & Si potenciómetro fuera de rango calibrado: saturación a límites definidos. \\
\bottomrule
\end{tabular}
\end{table}

\subsubsection{Caso de uso 3: Telemetría Bluetooth hacia app}

\begin{table}[h]
\centering
\begin{tabular}{ll}
\toprule
Elemento & Definición \\
\midrule
Disparador & Cambio de estado de luz o de porcentaje del potenciómetro. \\
Precondiciones & BT habilitado por DIP1, módulo HC-06 conectado. \\
Flujo básico & Firmware arma trama binaria de 2 bytes y transmite por USART1 para que la app informe el estado del sistema. \\
Alternativas & Si BT deshabilitado, no se transmite. \\
\bottomrule
\end{tabular}
\end{table}

\subsubsection{Caso de uso 4: Recuperación tras falla}

\begin{table}[h]
\centering
\begin{tabular}{ll}
\toprule
Elemento & Definición \\
\midrule
Disparador & Error de inicialización o forzado de \texttt{FAULT\} por DIP4 (\texttt{PA4\}). \\
Precondiciones & Sistema energizado. \\
Flujo básico & Corte de salidas de potencia, alarma visual/sonora según DIP, reintento de inicialización luego de timeout. \\
Alternativas & Si DIP4 vuelve a 0, salida de \texttt{FAULT\} y retorno a \texttt{NORMAL\}. \\
\bottomrule
\end{tabular}
\end{table}

Nota de trazabilidad de alcance:
\begin{itemize}
\item El informe de avances redefinió el alcance Bluetooth para visualización de estado (sin control remoto completo de actuadores).
\item Los casos de uso y la app se documentan en consecuencia: recepción de telemetría y presentación de estado.
\end{itemize}

\subsection{Descripción de módulos principales}

\subsubsection{2.3.1 Módulo de control (NUCLEO-F103RB)}
\begin{itemize}
\item Ejecuta scheduler cooperativo con tick de 1 ms.
\item Corre tres tareas: \texttt{task\textbackslash{}\_adc\}, \texttt{task\textbackslash{}\_system\}, \texttt{task\textbackslash{}\_pwm\}.
\end{itemize}

\subsubsection{2.3.2 Módulo de potencia (dimmer)}
\begin{itemize}
\item Dos canales de disparo TRIAC (luz y ventilador).
\item Optoacople de disparo y red de protección.
\end{itemize}

\subsubsection{2.3.3 Módulo de detección de cruce por cero (ZCD)}
\begin{itemize}
\item Entrada AC aislada y acondicionada a señal digital.
\item Entrada de interrupción por \texttt{PC2\} (EXTI).
\end{itemize}

\subsubsection{2.3.4 Módulo Bluetooth (HC-06)}
\begin{itemize}
\item Interfaz UART transparente en \texttt{PA9/PA10\}.
\item Configuración AT realizada con interfaz auxiliar USB-UART (Arduino).
\end{itemize}

\subsubsection{2.3.5 Aplicación móvil (MIT App Inventor)}
\begin{itemize}
\item Lectura de trama binaria de 2 bytes.
\item Visualización del porcentaje y estado de luz.
\end{itemize}

\hrulefill

\section{Diseño e implementación}

\subsection{Arquitectura general}

El sistema se organiza en dos dominios:
\begin{itemize}
\item Dominio lógico de 3.3 V (STM32 + entradas + comunicaciones).
\item Dominio de potencia AC (TRIAC + ZCD + protecciones).
\end{itemize}

\textbf{Figura 3.1 - Diagrama en bloques general\}
\begin{figure}[H]
\centering
\includegraphics[width=0.85\linewidth]{https://github.com/Embebidos-Fran-Marcos-Nacho/tdse-tf\_1-2/blob/663d795450e29c452e59a7ecae6f23108cb3e22d/Memoria\%20t\%C3\%A9cnica/imgs/diagrama\%20en\%20bloques.jpg}
\caption{Imagen}

\end{figure}

\textit{Epígrafe: Diagrama de bloques general.\}



\subsection{Diseño de hardware}

\subsubsection{3.2.1 Criterio de interconexión y montaje}

Se trabajó con placas y conexiones soldadas para la integración funcional final (sin protoboard ni cables Dupont en el montaje objetivo), en línea con las pautas de entrega.

Se usaron dos placas:
\begin{itemize}
\item placa shield para interfaz y conexión con NUCLEO.
\item placa dimmer para potencia, ZCD y protecciones.
\end{itemize}

\subsubsection{Etapa de conversión de niveles}

\textbf{Figura 3.2 - Esquemático del conversor de niveles\}
\begin{figure}[H]
\centering
\includegraphics[width=0.85\linewidth]{https://github.com/Embebidos-Fran-Marcos-Nacho/tdse-tf\_1-2/blob/c2fc7354b11ef4655cebe90b4b788acc5695045a/Memoria\%20t\%C3\%A9cnica/imgs/esquema\%20niveles.png}
\caption{Imagen}

\end{figure}
\textit{Epígrafe: Esquemático del conversor de niveles.\}

Se requirió para unir la placa F103RB (3.3 V) con la placa diseñada (5 V).

\subsubsection{Etapa de Triacs}

\textbf{Figura 3.3 - Esquemático de driver de TRIAC\}
\begin{figure}[H]
\centering
\includegraphics[width=0.85\linewidth]{https://github.com/Embebidos-Fran-Marcos-Nacho/tdse-tf\_1-2/blob/c2fc7354b11ef4655cebe90b4b788acc5695045a/Memoria\%20t\%C3\%A9cnica/imgs/esquem\%20triac.png}
\caption{Imagen}

\end{figure}
\textit{Epígrafe: Esquemático de driver de TRIAC.\}

Diseño tomado de las notas de aplicación que se encuentran en este mismo repositorio en la sección de hardware.

\subsubsection{3.2.2 Etapa ZCD (detección de cruce por cero)}

La etapa de ZCD fue validada progresivamente en banco antes de integrar potencia. Se observó que:
\begin{itemize}
\item la salida detectada requiere compensación temporal aproximada de 500 us para ubicar el cruce real.
\item las simulaciones resultaron consistentes con la tendencia medida.
\end{itemize}

\textbf{Figura 3.4 - Esquemático del ZCD\}

\begin{figure}[H]
\centering
\includegraphics[width=0.85\linewidth]{https://github.com/Embebidos-Fran-Marcos-Nacho/tdse-tf\_1-2/blob/c2fc7354b11ef4655cebe90b4b788acc5695045a/Memoria\%20t\%C3\%A9cnica/imgs/esquematico\%20ZCD.png}
\caption{Imagen}

\end{figure}
\textit{Epígrafe: Esquemático del ZCD.\}



\textbf{Figura 3.5 - Banco inicial de pruebas ZCD\}
\begin{figure}[H]
\centering
\includegraphics[width=0.85\linewidth]{https://github.com/Embebidos-Fran-Marcos-Nacho/tdse-tf\_1-2/blob/663d795450e29c452e59a7ecae6f23108cb3e22d/Memoria\%20t\%C3\%A9cnica/cosas\%20e\%20imagenes\%20para\%20memoria\%20t\%C3\%A9cnica\%20-\%20hardware/ZCD/banco\%20de\%20trabajo\%20inicial.jpeg}
\caption{Imagen}

\end{figure}
\textit{Epígrafe: Banco de trabajo durante las verificaciones del ZCD con osciloscopio.\}


\textbf{Figura 3.6 - Mediciones de pulsos ZCD (osciloscopio)\}
\begin{figure}[H]
\centering
\includegraphics[width=0.85\linewidth]{https://github.com/Embebidos-Fran-Marcos-Nacho/tdse-tf\_1-2/blob/663d795450e29c452e59a7ecae6f23108cb3e22d/Memoria\%20t\%C3\%A9cnica/cosas\%20e\%20imagenes\%20para\%20memoria\%20t\%C3\%A9cnica\%20-\%20hardware/ZCD/mediciones\%20pulsos.jpeg}
\caption{Imagen}

\end{figure}
\textit{Epígrafe: Pulsos de salida del ZCD - cursor midiendo tiempo entre pulsos.\}

Nótese que el ZCD actúa en cada cruce por cero, generando una señal de 100 Hz.

\textbf{Figura 3.7 - Medición de ancho de pulso del ZCD\}
\begin{figure}[H]
\centering
\includegraphics[width=0.85\linewidth]{https://github.com/Embebidos-Fran-Marcos-Nacho/tdse-tf\_1-2/blob/663d795450e29c452e59a7ecae6f23108cb3e22d/Memoria\%20t\%C3\%A9cnica/cosas\%20e\%20imagenes\%20para\%20memoria\%20t\%C3\%A9cnica\%20-\%20hardware/ZCD/mediciones\%20pulsos\%201.jpeg}
\caption{Imagen}

\end{figure}
\textit{Epígrafe: Salida del ZCD con la senoidal aplicada - cursor midiendo ancho de pulso.\}

\textbf{Figura 3.8 - Disparo previo al cruce real (senoidal negativa)\}
\begin{figure}[H]
\centering
\includegraphics[width=0.85\linewidth]{https://github.com/Embebidos-Fran-Marcos-Nacho/tdse-tf\_1-2/blob/663d795450e29c452e59a7ecae6f23108cb3e22d/Memoria\%20t\%C3\%A9cnica/cosas\%20e\%20imagenes\%20para\%20memoria\%20t\%C3\%A9cnica\%20-\%20hardware/ZCD/mediciones\%20pulsos\%202.jpeg}
\caption{Imagen}

\end{figure}
\textit{Epígrafe: Salida del ZCD con la senoidal aplicada - cursor midiendo tiempo de disparo previo al cruce por cero real con senoidal negativa.\}

\textbf{Figura 3.9 - Disparo previo al cruce real (senoidal positiva)\}
\begin{figure}[H]
\centering
\includegraphics[width=0.85\linewidth]{https://github.com/Embebidos-Fran-Marcos-Nacho/tdse-tf\_1-2/blob/663d795450e29c452e59a7ecae6f23108cb3e22d/Memoria\%20t\%C3\%A9cnica/cosas\%20e\%20imagenes\%20para\%20memoria\%20t\%C3\%A9cnica\%20-\%20hardware/ZCD/mediciones\%20pulsos\%204.jpeg}
\caption{Imagen}

\end{figure}
\textit{Epígrafe: Salida del ZCD con la senoidal aplicada - cursor midiendo tiempo de disparo previo al cruce por cero real con senoidal positiva.\}

El retardo fijo de disparo de los triacs se estimó tomando de referencia los tiempos de disparo del ZCD respecto del cruce real mostrados en estas imágenes.


\subsubsection{3.2.3 Etapa de potencia y protecciones}

Según esquemático principal (\texttt{Hardware/placa dimmer/dimmer.kicad\textbackslash{}\_sch\}), el canal de potencia integra:
\begin{itemize}
\item TRIAC de potencia (\texttt{BTA06-600C\}).
\item Optoacoplador de disparo (\texttt{MOC3023M\}).
\item Elementos de protección (varistor, fusible, red RC/snubber opcional).
\end{itemize}

Notas de fabricación y prueba:
\begin{itemize}
\item Primero se validó el correcto funcionamiento del ZCD, luego se integraron TRIACs.
\item Las primeras pruebas integradas se hicieron en 24 VAC. Esto conllevó una ligera y reversible modificación del circuito de ZCD.
\end{itemize}

\textbf{Figura 3.10 - Ensayo de salida de optoacoplador\}
\begin{figure}[H]
\centering
\includegraphics[width=0.85\linewidth]{https://github.com/Embebidos-Fran-Marcos-Nacho/tdse-tf\_1-2/blob/663d795450e29c452e59a7ecae6f23108cb3e22d/Memoria\%20t\%C3\%A9cnica/cosas\%20e\%20imagenes\%20para\%20memoria\%20t\%C3\%A9cnica\%20-\%20hardware/ZCD/salida\%20real\%20del\%20opto.jpeg}
\caption{Imagen}

\end{figure}
\textit{Epígrafe: Señal a la salida del 4N25 en configuración de emisor común/negador.\}

\textbf{Figura 3.11 - Simulación de ZCD y salida de opto\}
\begin{figure}[H]
\centering
\includegraphics[width=0.85\linewidth]{https://github.com/Embebidos-Fran-Marcos-Nacho/tdse-tf\_1-2/blob/663d795450e29c452e59a7ecae6f23108cb3e22d/Memoria\%20t\%C3\%A9cnica/cosas\%20e\%20imagenes\%20para\%20memoria\%20t\%C3\%A9cnica\%20-\%20hardware/ZCD/simu\%20ZCD\%20proper.jpeg}
\caption{Imagen}

\end{figure}
\textit{Epígrafe: Simulación de la entrada y salida ideal del ZCD.\}

Nótese que es muy parecida a la medida.

\textbf{Figura 3.12 - Salida simulada del 4N25\}
\begin{figure}[H]
\centering
\includegraphics[width=0.85\linewidth]{https://github.com/Embebidos-Fran-Marcos-Nacho/tdse-tf\_1-2/blob/663d795450e29c452e59a7ecae6f23108cb3e22d/Memoria\%20t\%C3\%A9cnica/cosas\%20e\%20imagenes\%20para\%20memoria\%20t\%C3\%A9cnica\%20-\%20hardware/ZCD/simu\%20salida\%20del\%20optoacoplador.jpeg}
\caption{Imagen}

\end{figure}
\textit{Epígrafe: Salida simulada del 4N25.\}

No se parece mucho a la real, pero funcionó igual: la tensión alcanzó el umbral para disparar los Schmitt triggers.


\subsubsection{3.2.4 Fabricación de placas}

Se documentó el proceso de fabricación con transferencia y ataque químico:
\begin{itemize}
\item Primero se imprimió el diseño sobre un papel PnP Blue.
\item Luego se transfirió por medio de calor.
\item Se hicieron las correcciones manuales de transferencia.
\item Por último, se realizó un control de continuidad previo a energizar.
\end{itemize}

Lecciones aprendidas para próxima iteración:
\begin{itemize}
\item Revisar diámetros de agujeros para componentes de potencia (varistores y componentes grandes).
\item Simplificar topología de ZCD.
\item Evaluar integración de control de dimming en una etapa dedicada.
\end{itemize}

\textbf{Figura 3.13 - Papel de transferencia con diseño impreso\}
\begin{figure}[H]
\centering
\includegraphics[width=0.85\linewidth]{https://github.com/Embebidos-Fran-Marcos-Nacho/tdse-tf\_1-2/blob/1030475e09d21a3204b19eb7996e9f11bb688033/Memoria\%20t\%C3\%A9cnica/cosas\%20e\%20imagenes\%20para\%20memoria\%20t\%C3\%A9cnica\%20-\%20hardware/fab\%20placa/p\%20n\%20p\%20blue.jpeg}
\caption{Imagen}

\end{figure}
\textit{Epígrafe: Papel de transferencia con el diseño impreso.\}

\textbf{Figura 3.14 - Transferencia previa a correcciones\}
\begin{figure}[H]
\centering
\includegraphics[width=0.85\linewidth]{https://github.com/Embebidos-Fran-Marcos-Nacho/tdse-tf\_1-2/blob/1030475e09d21a3204b19eb7996e9f11bb688033/Memoria\%20t\%C3\%A9cnica/cosas\%20e\%20imagenes\%20para\%20memoria\%20t\%C3\%A9cnica\%20-\%20hardware/fab\%20placa/trasferencia\%20a\%20cobre.jpeg}
\caption{Imagen}

\end{figure}
\textit{Epígrafe: Transferencia previa a correcciones.\}

\textbf{Figura 3.15 - Transferencia corregida\}
\begin{figure}[H]
\centering
\includegraphics[width=0.85\linewidth]{https://github.com/Embebidos-Fran-Marcos-Nacho/tdse-tf\_1-2/blob/1030475e09d21a3204b19eb7996e9f11bb688033/Memoria\%20t\%C3\%A9cnica/cosas\%20e\%20imagenes\%20para\%20memoria\%20t\%C3\%A9cnica\%20-\%20hardware/fab\%20placa/correci\%C3\%B3n\%20de\%20desperfectos\%20de\%20trasnferencia.jpeg}
\caption{Imagen}

\end{figure}
\textit{Epígrafe: Transferencia corregida.\}

\textbf{Figura 3.16 - Placa fabricada\}
\begin{figure}[H]
\centering
\includegraphics[width=0.85\linewidth]{https://github.com/Embebidos-Fran-Marcos-Nacho/tdse-tf\_1-2/blob/663d795450e29c452e59a7ecae6f23108cb3e22d/Memoria\%20t\%C3\%A9cnica/cosas\%20e\%20imagenes\%20para\%20memoria\%20t\%C3\%A9cnica\%20-\%20hardware/fab\%20placa/cobre\%20etched.jpeg}
\caption{Imagen}

\end{figure}
\textit{Epígrafe: Placa fabricada.\}


\subsubsection{3.2.5 Pinout del sistema (STM32F103RB)}

\begin{table}[h]
\centering
\begin{tabular}{ll}
\toprule
Pin & Función \\
\midrule
\texttt{PA0\} & Potenciómetro (ADC) \\
\texttt{PC0\} & DIP1: habilitación Bluetooth \\
\texttt{PC1\} & DIP2: habilitación buzzer \\
\texttt{PB0\} & DIP3: habilitación LED \\
\texttt{PA4\} & DIP4: forzado de estado \texttt{FAULT\} \\
\texttt{PC12\} & Botón ON de luz \\
\texttt{PC9\} & Botón OFF de luz \\
\texttt{PC2\} & ZCD (EXTI) \\
\texttt{PB3\} & TRIAC canal ventilador \\
\texttt{PB4\} & TRIAC canal luz \\
\texttt{PB13\} & LED \\
\texttt{PA8\} & Buzzer (\texttt{TIM1\textbackslash{}\_CH1\}) \\
\texttt{PA9/PA10\} & USART1 (HC-06) \\
\texttt{PA2/PA3\} & USART2 (consola ST-Link VCP) \\
\texttt{PC8\} & Onda de prueba 100 Hz (modo test) \\
\bottomrule
\end{tabular}
\end{table}

\subsubsection{3.2.6 Cableado e imágenes del montaje}

\textbf{Figura 3.17 - Cableado final del prototipo\}

\begin{figure}[H]
\centering
\includegraphics[width=0.85\linewidth]{https://github.com/Embebidos-Fran-Marcos-Nacho/tdse-tf\_1-2/blob/3cb04d32ab982e06ec97e47ec6184a648ebf46cf/Memoria\%20t\%C3\%A9cnica/cosas\%20e\%20imagenes\%20para\%20memoria\%20t\%C3\%A9cnica\%20-\%20hardware/banco\%20de\%20trabajo\%20desprolijo/banco\%20final.jpeg}
\caption{Imagen}

\end{figure}

\textit{Epígrafe: Montaje final del prototipo durante ensayo integrado.\}


\textbf{Figura 3.18 - Diagrama de conexión entre placas simplificado\}

\begin{figure}[H]
\centering
\includegraphics[width=0.85\linewidth]{https://github.com/Embebidos-Fran-Marcos-Nacho/tdse-tf\_1-2/blob/00693ac864a65b0389699a47c52606a88d0adbb9/Diagrama\%20de\%20conexi\%C3\%B3n\%20simplificado/conexionado.png}
\caption{Imagen}

\end{figure}

\textit{Epígrafe: Diagrama simplificado de conexión entre placas.\}

\textbf{Figura 3.19 - Overview de placa shield y conexionado\}

\begin{figure}[H]
\centering
\includegraphics[width=0.85\linewidth]{https://github.com/Embebidos-Fran-Marcos-Nacho/tdse-tf\_1-2/blob/c2fc7354b11ef4655cebe90b4b788acc5695045a/Diagrama\%20de\%20conexi\%C3\%B3n\%20simplificado/f103rb.jpg}
\caption{Imagen}

\end{figure}

\textit{Epígrafe: Vista general y conexionado de la shield para F103RB.\}

\textbf{Figura 3.20 - Conexionado de placa de TRIACs\}

\begin{figure}[H]
\centering
\includegraphics[width=0.85\linewidth]{https://github.com/Embebidos-Fran-Marcos-Nacho/tdse-tf\_1-2/blob/c2fc7354b11ef4655cebe90b4b788acc5695045a/Diagrama\%20de\%20conexi\%C3\%B3n\%20simplificado/triacs.jpg}
\caption{Imagen}

\end{figure}

\textit{Epígrafe: Conexionado de la placa de TRIACs y cargas.\}


\subsection{Diseño de firmware}

\subsubsection{3.3.1 Arquitectura de ejecución}

El firmware implementa un esquema \textit{bare-metal\} con super-loop y tick de 1 ms (\texttt{HAL\textbackslash{}\_SYSTICK\textbackslash{}\_Callback\}), recorriendo en orden fijo:
\begin{enumerate}
\item \texttt{task\textbackslash{}\_adc\textbackslash{}\_update\}
\item \texttt{task\textbackslash{}\_system\textbackslash{}\_update\}
\item \texttt{task\textbackslash{}\_pwm\textbackslash{}\_update\}
\end{enumerate}

Cada tarea se ejecuta en cada tick y su tiempo se mide con contador de ciclos (\texttt{DWT->CYCCNT\}) para cálculo de WCET.

\subsubsection{3.3.2 Máquina de estados del sistema}

\texttt{task\textbackslash{}\_system.c\} implementa la máquina de estado global:
\begin{itemize}
\item \texttt{ST\textbackslash{}\_INIT\textbackslash{}\_READ\textbackslash{}\_FLASH\}
\item \texttt{ST\textbackslash{}\_INIT\textbackslash{}\_READ\textbackslash{}\_DIP\}
\item \texttt{ST\textbackslash{}\_INIT\textbackslash{}\_CHECK\textbackslash{}\_SENSORS\}
\item \texttt{ST\textbackslash{}\_INIT\textbackslash{}\_RESTORE\textbackslash{}\_PWM\}
\item \texttt{ST\textbackslash{}\_INIT\textbackslash{}\_CONFIG\textbackslash{}\_BT\}
\item \texttt{ST\textbackslash{}\_NORMAL\}
\item \texttt{ST\textbackslash{}\_FAULT\}
\end{itemize}

En \texttt{FAULT\}:
\begin{itemize}
\item se corta potencia (\texttt{cut\textbackslash{}\_off\textbackslash{}\_voltage=true\}).
\item se activa patrón de alarma.
\item se reintenta inicialización por timeout.
\end{itemize}

\textbf{Figura 3.21 - Statechart general (Harel/Itemis)\}
\begin{figure}[H]
\centering
\includegraphics[width=0.85\linewidth]{https://github.com/Embebidos-Fran-Marcos-Nacho/tdse-tf\_1-2/blob/3cb04d32ab982e06ec97e47ec6184a648ebf46cf/Memoria\%20t\%C3\%A9cnica/imgs/Statechart.png}
\caption{Imagen}

\end{figure}

\textbf{Figura 3.22 - Subestados de inicialización\}
\begin{figure}[H]
\centering
\includegraphics[width=0.85\linewidth]{https://github.com/Embebidos-Fran-Marcos-Nacho/tdse-tf\_1-2/blob/169b5aabae5e8c5a7af391914271a01397db4f61/Memoria\%20t\%C3\%A9cnica/imgs/State\%20Init.png}
\caption{Imagen}

\end{figure}

\textbf{Figura 3.23 - Estado normal\}
\begin{figure}[H]
\centering
\includegraphics[width=0.85\linewidth]{https://github.com/Embebidos-Fran-Marcos-Nacho/tdse-tf\_1-2/blob/3cb04d32ab982e06ec97e47ec6184a648ebf46cf/Memoria\%20t\%C3\%A9cnica/imgs/State\%20Normal.png}
\caption{Imagen}

\end{figure}

\textbf{Figura 3.24 - Estado de falla\}
\begin{figure}[H]
\centering
\includegraphics[width=0.85\linewidth]{https://github.com/Embebidos-Fran-Marcos-Nacho/tdse-tf\_1-2/blob/3cb04d32ab982e06ec97e47ec6184a648ebf46cf/Memoria\%20t\%C3\%A9cnica/imgs/State\%20Fault\_ST.png}
\caption{Imagen}

\end{figure}

\textbf{Figura 3.25 - FSM de debounce de botón\}
\begin{figure}[H]
\centering
\includegraphics[width=0.85\linewidth]{https://github.com/Embebidos-Fran-Marcos-Nacho/tdse-tf\_1-2/blob/3cb04d32ab982e06ec97e47ec6184a648ebf46cf/Memoria\%20t\%C3\%A9cnica/imgs/ST\_BTN.png}
\caption{Imagen}

\end{figure}

\subsubsection{3.3.3 Entradas y acondicionamiento lógico}

\begin{itemize}
\item Debounce por máquina de estados para botones ON/OFF.
\item Muestreo ADC periódico (\texttt{ADC\textbackslash{}\_PERIOD\textbackslash{}\_MS = 50 ms\}).
\item Escalado del potenciómetro usando límites de calibración manual:
\item mínimo: 696 cuentas.
\item máximo: 3194 cuentas.
\item Filtro por deadband para evento de potenciómetro (\texttt{APP\textbackslash{}\_ADC\textbackslash{}\_PERCENT\textbackslash{}\_EVENT\textbackslash{}\_DEADBAND = 2\textbackslash{}\%\}) para evitar oscilaciones por ruido (ej. 99\% <-> 100\%).
Esto último asegura una excursión correcta que considera las caidas de tensión en la placa de control.
\end{itemize}

\subsubsection{3.3.4 Control de TRIAC y sincronización AC}

\texttt{task\textbackslash{}\_pwm.c\} usa \texttt{TIM2\} para programar ventanas ON/OFF por semiciclo:
\begin{itemize}
\item retardo fijo de referencia: \texttt{APP\textbackslash{}\_TRIAC\textbackslash{}\_FIXED\textbackslash{}\_WAIT\textbackslash{}\_US = 700 us\}.
\item ancho de pulso de gate: \texttt{APP\textbackslash{}\_TRIAC\textbackslash{}\_PULSE\textbackslash{}\_US = 1000 us\}.
\item retardo variable del ventilador por porcentaje (\texttt{fan\textbackslash{}\_delay\textbackslash{}\_us\}).
\end{itemize}

El evento de cruce por cero llega por EXTI en \texttt{PC2\}.

\subsubsection{3.3.5 Persistencia en flash}

Se utiliza una página dedicada de flash interna (\texttt{0x0801FC00\}) para:
\begin{itemize}
\item palabra mágica.
\item versión de layout.
\item estado de luz.
\item calibración ADC min/max.
\end{itemize}

Si el guardado crítico falla (según configuración estricta), la FSM puede entrar en \texttt{FAULT\}.

\subsubsection{3.3.6 Bluetooth HC-06}

Configuración:
\begin{itemize}
\item nombre: \texttt{Dimmer\textbackslash{}\_BL\}.
\item PIN: \texttt{1111\}.
\item comandos AT enviados sin CR/LF y con retardos adecuados.
\end{itemize}

Funcionamiento en firmware:
\begin{itemize}
\item UART por \texttt{USART1\}.
\item telemetría binaria (sin JSON).
\item 2 bytes por frame:
\item byte 0: \texttt{adc\textbackslash{}\_percent\} (0..100).
\item byte 1: \texttt{light\textbackslash{}\_enabled\} (0/1).
\item Envío periódico por tiempo (no por cambio), configurable con \texttt{APP\textbackslash{}\_BT\textbackslash{}\_TELEMETRY\textbackslash{}\_PERIOD\textbackslash{}\_MS\} (actualmente \texttt{50 ms\}). Esto ayudó mucho a mejorar los WCET debido a que el uso de la consola parece tomar mucho tiempo.
\end{itemize}

Nota: actualmente la app se usa como receptor de estado, no como control remoto completo de actuadores.

\subsubsection{3.3.7 Aplicación móvil}

La app fue desarrollada en MIT App Inventor. Se documentan interfaz y bloques de procesamiento de bytes.

\textbf{Figura 3.26 - Pantalla principal app\}

\begin{figure}[H]
\centering
\includegraphics[width=0.85\linewidth]{https://github.com/Embebidos-Fran-Marcos-Nacho/tdse-tf\_1-2/blob/566a7314061481abbec17f240388ee198cea82ee/Memoria\%20t\%C3\%A9cnica/cosas\%20e\%20imagenes\%20para\%20memoria\%20t\%C3\%A9cnica\%20-\%20hardware/captura\%20app.jpeg}
\caption{Imagen}

\end{figure}

\textit{Epígrafe: Pantalla principal de la App.\}


\textbf{Figura 3.27 - Bloques MIT App Inventor (parte 1)\}

\begin{figure}[H]
\centering
\includegraphics[width=0.85\linewidth]{https://github.com/Embebidos-Fran-Marcos-Nacho/tdse-tf\_1-2/blob/65b6a1be5b7a1b68e959d041707e17e00ebe5659/Memoria\%20t\%C3\%A9cnica/imgs/mit\%20app\%20bloque\%201.png}
\caption{Imagen}

\end{figure}

\textit{Epígrafe: Bloques de inicialización de la pantalla principal.\}

\textbf{Figura 3.28 - Bloques MIT App Inventor (parte 2)\}

\begin{figure}[H]
\centering
\includegraphics[width=0.85\linewidth]{https://github.com/Embebidos-Fran-Marcos-Nacho/tdse-tf\_1-2/blob/65b6a1be5b7a1b68e959d041707e17e00ebe5659/Memoria\%20t\%C3\%A9cnica/imgs/mit\%20app\%20bloque\%202.png}
\caption{Imagen}

\end{figure}

\textit{Epígrafe: Lógica de actualización de datos y pantalla.\}

\textbf{Figura 3.29 - Bloques MIT App Inventor (parte 3)\}

\begin{figure}[H]
\centering
\includegraphics[width=0.85\linewidth]{https://github.com/Embebidos-Fran-Marcos-Nacho/tdse-tf\_1-2/blob/65b6a1be5b7a1b68e959d041707e17e00ebe5659/Memoria\%20t\%C3\%A9cnica/imgs/mit\%20app\%20bloque\%203.png}
\caption{Imagen}

\end{figure}

\textit{Epígrafe: Lógica de selección de dispositivo bluetooth.\}

\hrulefill

\section{Ensayos y resultados}

\subsection{Pruebas funcionales de hardware}

\begin{table}[h]
\centering
\begin{tabularx}{\linewidth}{llX}
\toprule
Ensayo & Resultado & Estado \\
\midrule
Integridad de placas (continuidad) & Validación previa a energización & ✅ \\
ZCD en banco & Detección de eventos y correlación con simulación & ✅ \\
Integración con 24 VAC & Prueba inicial de etapa integrada & ✅ \\
Observar integridad de dimming en 24 VAC (osciloscopio) & Se verificó por medio de osciloscopio & ✅ \\
\bottomrule
\end{tabularx}
\end{table}

\subsection{Pruebas funcionales de firmware}

\begin{table}[h]
\centering
\begin{tabularx}{\linewidth}{llX}
\toprule
Ensayo & Resultado & Estado \\
\midrule
Debounce botones ON/OFF & Eventos limpios sobre FSM & ✅ \\
Muestreo ADC + mapeo & Escalado operativo 0..100\% & ✅ \\
FSM de sistema (\texttt{INIT/NORMAL/FAULT\}) & Transiciones válidas en logs & ✅ \\
Persistencia flash & Lectura/escritura de estado y calibración & ✅ \\
Telemetría BT (2 bytes) & Trama enviada en forma periódica (\texttt{APP\textbackslash{}\_BT\textbackslash{}\_TELEMETRY\textbackslash{}\_PERIOD\textbackslash{}\_MS\}) & ✅ \\
\bottomrule
\end{tabularx}
\end{table}

\subsection{Pruebas de integración}

Se validó la interacción completa:
\begin{itemize}
\item entradas físicas.
\item control de potencia.
\item telemetría hacia app.
\end{itemize}

\textbf{Video de integración en funcionamiento\}

\url{https://youtu.be/iv2bGrqrMtU}



\subsection{Medición y análisis de consumo}

Metodología aplicada:
\begin{itemize}
\item medición de consumo total en la entrada de \texttt{5V\} del sistema (NUCLEO + shield).
\item alimentación desde fuente externa conectada a pines \texttt{5V\} y \texttt{GND\}.
\item medición de corriente con multímetro en serie sobre la línea de \texttt{5V\}.
\item medición de tensión en bornes de entrada para estimar potencia (\texttt{P = V * I\}).
\end{itemize}

Procedimiento realizado:
\begin{enumerate}
\item Desconectar USB/ST-Link para evitar doble alimentación.
\item Conectar fuente externa a \texttt{5V\} y \texttt{GND\}.
\item Ajustar la fuente para garantizar \texttt{5V\} en el pin \texttt{5V\} de la placa (compensando caídas en cables).
\item Intercalar amperímetro en serie en la línea de \texttt{5V\}.
\item Medir tensión de entrada en paralelo sobre \texttt{5V-GND\}.
\item Registrar datos en los modos:
\end{enumerate}
\begin{itemize}
\item normal sin módulo Bluetooth conectado.
\item normal con módulo Bluetooth conectado pero desactivado.
\item normal con Bluetooth activo enviando datos.
\item fault con alarma activa (buzzer + LED).
\end{itemize}
\begin{enumerate}
\item Debido a que el consumo oscila rápidamente en el tiempo, se tomó como referencia el valor pico observado en cada modo.
\end{enumerate}

Alcance de la medición:
\begin{itemize}
\item Esta medición representa el consumo total a \texttt{5V\} del conjunto montado.
\item El riel de \texttt{3.3V\} queda incluido indirectamente, ya que se genera desde \texttt{5V\} mediante el regulador de la placa. Además, registrar el consumo de 3.3V solo no tiene sentido para un sistema que se alimenta com 5V.
\end{itemize}

\begin{table}[h]
\centering
\begin{tabularx}{\linewidth}{lrrX}
\toprule
Modo & I pico @5V [mA] & P pico @5V [W] & Observaciones \\
\midrule
Normal sin módulo BT (desconectado) & 64 & 0.320 & Escenario de menor consumo; representa una forma válida de uso sin telemetría Bluetooth. \\
Normal con módulo BT conectado y desactivado & 104 & 0.520 & Aumento de consumo por presencia/alimentación del módulo Bluetooth. \\
Normal con BT activo enviando datos & 107 & 0.535 & Incremento leve respecto al modo BT desactivado. \\
Fault (buzzer + LED activos) & 145 & 0.725 & Peor caso medido en operación. \\
\bottomrule
\end{tabularx}
\end{table}

Análisis:
\begin{itemize}
\item Potencia calculada como \texttt{P = V * I\}, usando \texttt{V = 5V\} y corriente pico medida en cada modo.
\item El peor caso medido fue \texttt{145 mA\} a \texttt{5V\}, equivalente a \texttt{0.725 W\}.
\item El sistema se mantiene por debajo de \texttt{1 W\}, por lo que puede alimentarse sin inconvenientes con fuentes comerciales 220VAC->5V de baja potencia.
\item La diferencia entre BT desactivado y BT transmitiendo (\texttt{104 mA\} -> \texttt{107 mA\}) es baja, consistente con carga adicional moderada por comunicación.
\end{itemize}

\subsection{Console and Build Analyzer}

Resultado consolidado de herramientas de análisis de consola y build.

\textbf{Figura 4.1 - Console and Build Analyzer\}

\begin{figure}[H]
\centering
\includegraphics[width=0.85\linewidth]{https://github.com/Embebidos-Fran-Marcos-Nacho/tdse-tf\_1-2/blob/c2fc7354b11ef4655cebe90b4b788acc5695045a/Memoria\%20t\%C3\%A9cnica/imgs/build\%20console\%20y\%20analyzer.png}
\caption{Imagen}

\end{figure}

\textit{Epígrafe: Build console y build analyzer. Dice RAM: 10.31\textbackslash{}\% y FLASH: 16.11\textbackslash{}\% \}



\subsection{Medición y análisis de WCET por tarea}

El firmware instrumenta WCET por tarea en \texttt{app.c\} usando \texttt{DWT->CYCCNT\} y un modo de perfilado limpio (\texttt{[PROF]\}) activado temporalmente durante ensayo:
\begin{itemize}
\item \texttt{WCETw\} = WCET en ventana (steady-state, últimos 1000 ciclos)
\item \texttt{WCETb\} = WCET acumulado desde boot
\item \texttt{Cavg\} = tiempo promedio de ejecución
\end{itemize}

Metodología realizada:
\begin{enumerate}
\item Flashear build \texttt{Software STM32/main\} en NUCLEO-F103RB.
\item Abrir consola serial (USART2, 115200 baud).
\item Ejecutar con trazas de test desactivadas (\texttt{APP\textbackslash{}\_TEST\textbackslash{}\_MODE = 0\}) y perfil limpio activo durante la medición.
\item Dejar correr el sistema en estado idle (sin pulsaciones ni cambios ADC).
\item Registrar múltiples ventanas \texttt{[PROF]\} (n\textasciitilde{}1010 por ventana).
\end{enumerate}

Es decir, se usó el mismo programa pero con un modo de estimación del WCET.

Formato de log utilizado y significado de parámetros:
\begin{itemize}
\item \texttt{n\}: cantidad de ciclos de scheduler medidos en la ventana.
\item \texttt{ov\}: cantidad de overruns (ciclos cuyo runtime total supera 1 ms).
\item \texttt{qmax\}: máximo backlog observado en la cola de ticks (\texttt{g\textbackslash{}\_app\textbackslash{}\_tick\textbackslash{}\_cnt\}) durante la ventana.
\item \texttt{Cavg=\textbackslash{}\{adc,sys,pwm\textbackslash{}\}\}: tiempo promedio por tarea en la ventana (us).
\item \texttt{WCETw=\textbackslash{}\{adc,sys,pwm\textbackslash{}\}\}: peor tiempo por tarea dentro de la ventana (us).
\item \texttt{CPU=\textbackslash{}\{avg,peak\textbackslash{}\}\}: utilización total promedio y pico del scheduler en la ventana (\%).
\item \texttt{U=\textbackslash{}\{avg,wcet\textbackslash{}\}\}: factor de uso promedio y por peor caso reportado para la ventana.
\end{itemize}

Criterio de consolidación de resultados:
\begin{itemize}
\item Se tomaron 15 líneas consecutivas \texttt{[PROF]\}.
\item Para \texttt{Cavg típico\} se reportó el rango estable observado.
\item Para \texttt{WCETw máx observado\} se tomó el máximo absoluto entre las 15 ventanas.
\item Para \texttt{U\} se reportó rango observado por ventana y cota conservadora adicional.
\end{itemize}

Es muy importante destacar que el uso de la consola eleva masivamente los WCET, por lo que se minimizó en las evaluaciones.

\textbf{Resultados medidos (estado idle/estable, 15 ventanas):\}

\begin{table}[h]
\centering
\begin{tabularx}{\linewidth}{lrrX}
\toprule
Tarea & Período asumido [us] & Cavg típico [us] & WCETw máx observado [us] \\
\midrule
\texttt{task\textbackslash{}\_adc\textbackslash{}\_update\} & 1000 & 64..66 & 268 \\
\texttt{task\textbackslash{}\_system\textbackslash{}\_update\} & 1000 & 26 & 125 \\
\texttt{task\textbackslash{}\_pwm\textbackslash{}\_update\} & 1000 & 46..48 & 292 \\
\bottomrule
\end{tabularx}
\end{table}

\textbf{Observaciones:\}
\begin{itemize}
\item No se observaron overruns (\texttt{ov=0\}) en ninguna ventana.
\item \texttt{qmax=10\} se mantuvo estable en todas las ventanas registradas.
\item Uso de CPU: \texttt{CPU avg\} entre \texttt{13.6\textbackslash{}\%\} y \texttt{14.0\textbackslash{}\%\}; \texttt{CPU peak\} entre \texttt{35.6\textbackslash{}\%\} y \texttt{38.0\textbackslash{}\%\}.
\end{itemize}


\subsection{Cálculo del factor de uso de CPU (U)}

Para evaluar la carga temporal del sistema se calculó el factor de utilización de CPU utilizando la expresión clásica de sistemas en tiempo real:

\textbackslash{}\textbackslash{}[U = \textbackslash{}sum\_\{i=1\}\textasciicircum{}\{n\} \textbackslash{}frac\{C\_i\}\{T\_i\}\textbackslash{}\textbackslash{}]

donde (C\_i) representa el WCET de la tarea (i), medido a partir de ventanas de ejecución en régimen estacionario, y (T\_i) su período de activación.

La Tabla siguiente resume los valores utilizados para el cálculo:

\begin{table}[h]
\centering
\begin{tabularx}{\linewidth}{lrrX}
\toprule
Tarea & (C\_i) (WCET) [µs] & (T\_i) [µs] & (C\_i/T\_i) \\
\midrule
\texttt{task\textbackslash{}\_adc\textbackslash{}\_update\} & 268 & 1000 & 0.268 \\
\texttt{task\textbackslash{}\_system\textbackslash{}\_update\} & 125 & 1000 & 0.125 \\
\texttt{task\textbackslash{}\_pwm\textbackslash{}\_update\} & 292 & 1000 & 0.292 \\
\textbf{Total (U) (WCET-based, conservador)\} & – & – & \textbf{0.685\} \\
\bottomrule
\end{tabularx}
\end{table}

El valor total obtenido, (\$U = 0.685\$), corresponde a una cota conservadora, ya que se construyó combinando los máximos tiempos de ejecución observados para cada tarea en ventanas temporales distintas y no a partir de una ocurrencia simultánea real de dichos máximos.

En contraste, las mediciones experimentales mostraron valores de utilización sensiblemente menores: la utilización basada en ventanas ((\$U\_\{wcet\}\$)) se mantuvo entre \$46,5\$ \% y \$66,1\$ \%, mientras que la utilización promedio ((\$U\_\{avg\}\$)) se ubicó en torno al 14 \% en régimen permanente. En el caso particular del STM32F103RB, estos resultados indican un comportamiento temporal estable, con un margen de CPU suficiente para absorber variaciones transitorias de ejecución sin comprometer el cumplimiento de los períodos de las tareas, validando así la factibilidad temporal del diseño.


\subsection{Gestión de bajo consumo y justificación}

En esta iteración del TP no se implementó una estrategia dedicada de bajo consumo a nivel firmware (por ejemplo, entrada explícita a modos \texttt{Sleep/Stop\} ni escalado dinámico de frecuencia), ya que el objetivo principal fue priorizar robustez funcional, seguridad eléctrica y cierre de integración.

No obstante, se evaluó el impacto energético real del sistema y los resultados muestran que el consumo del conjunto está dominado principalmente por el hardware periférico y la plataforma de prototipado:
\begin{itemize}
\item El salto de consumo al conectar el módulo Bluetooth es significativo (\texttt{64 mA\} -> \texttt{104 mA\}), aun sin transmitir.
\item La diferencia entre Bluetooth desactivado y transmitiendo es menor (\texttt{104 mA\} -> \texttt{107 mA\}).
\item En falla, el mayor consumo se explica por actuadores/indicadores (\texttt{buzzer + LED\}), no por carga computacional del CPU.
\end{itemize}

Esto es consistente con el factor de uso medido (\texttt{Uavg\} alrededor de \texttt{14\textbackslash{}\%\} y cota conservadora \texttt{Uwcet = 0.685\}): la carga temporal del microcontrolador no aparece como cuello de botella energético principal en el prototipo actual.

En una versión orientada a producto (placa dedicada, sin sobrecarga de NUCLEO y periféricos de laboratorio), sí corresponde aplicar optimización sistemática de consumo:

\begin{itemize}
\item Reducir frecuencia de reloj del MCU al mínimo compatible con temporización y control de TRIAC.
\item Incorporar política de idle de bajo consumo (entrada a \texttt{Sleep\} entre eventos periódicos/interrupts).
\item Migrar de HC-06 (Bluetooth clásico) a BLE para telemetría de bajo consumo.
\item Revisar arquitectura de hardware auxiliar (drivers, conversores, etapas de acondicionamiento y protecciones) para eliminar consumo no esencial.
\end{itemize}

Conclusión: para el alcance académico de esta entrega, el consumo observado está mayormente determinado por decisiones de hardware e instrumentación de prototipo. La optimización fina de bajo consumo queda planificada como mejora de próxima revisión de diseño.

\subsection{Cumplimiento de requisitos}

\begin{table}[h]
\centering
\begin{tabularx}{\linewidth}{llccX}
\toprule
ID & Requisito (versión final) & Hardware & Software & Estado final \\
\midrule
1.1 & El sistema permitirá encender y apagar las luces mediante un botón físico. & 🟢 & 🟢 & ✅ \\
1.2 & El sistema permitirá ajustar la velocidad del ventilador mediante un potenciómetro. & 🟢 & 🟢 & ✅ \\
1.3 & El sistema permitirá ver el estado del ventilador y las luces vía Bluetooth. & 🟢 & 🟢 & ✅ \\
2.1 & El sistema contará con un DIP switch para habilitar o deshabilitar el Bluetooth. & 🟢 & 🟢 & ✅ \\
2.2 & El DIP switch permitirá seleccionar configuraciones o canales del módulo Bluetooth. & 🟢 & 🔴 & 🔴 \\
3.1 & El sistema contará con LEDs que indiquen el estado del Bluetooth. & 🟢 & 🟢 & ✅ \\
3.2 & El sistema contará con un buzzer para señalizar eventos del sistema. & 🟢 & 🟢 & ✅ \\
4.1 & El sistema deberá guardar en memoria flash el último valor de PWM utilizado. & 🟢 & 🟢 & ✅ \\
4.2 & El sistema deberá restaurar automáticamente el último valor guardado al encender. & 🟢 & 🟢 & ✅ \\
5.1 & El sistema deberá operar de forma segura sobre cargas de 220 VAC. & 🟡 & N/A & 🟡 \\
6.1 & La aplicación dará información sobre los estados disponibles, que incluyen la velocidad del ventilador y el estado de luces. & N/A & 🟢 & ✅ \\
6.2 & El sistema deberá evitar conflictos entre el control físico y la comunicación Bluetooth, incluyendo conflictos de timings. & N/A & 🟢 & ✅ \\
\bottomrule
\end{tabularx}
\end{table}

Leyenda:
\begin{itemize}
\item 🟢 implementado
\item 🟡 parcialmente cumplido / con alcance acotado en prototipo
\item 🔴 no implementado / descartado
\item ✅ cumplido
\end{itemize}

Observación sobre el requisito 5.1 (220 VAC):
\begin{itemize}
\item La validación final sobre red de 220 VAC queda planificada para la etapa posterior a la aprobación académica del trabajo.
\item Esta decisión se toma para reducir el riesgo de daño de la placa durante la instancia de entrega y evaluación.
\end{itemize}

Observación sobre el requisito 2.2 (canales/configuración Bluetooth):
\begin{itemize}
\item En la implementación final no se desarrolló la selección de canales/configuraciones por DIP para Bluetooth.
\item Se descartó por no ser necesario para el funcionamiento objetivo del sistema (telemetría de estado).
\end{itemize}

\subsection{Comparación con sistemas similares}

\begin{table}[h]
\centering
\begin{tabularx}{\linewidth}{lccX}
\toprule
Característica & Control IR/RF básico & Solución Wi-Fi comercial & Este proyecto \\
\midrule
Interfaz local de pared & No & Generalmente no & Sí \\
App móvil & No & Sí & Sí (telemetría) \\
Personalización firmware & No & No & Sí \\
Persistencia local & Variable & Sí & Sí \\
Costo de prototipo académico & N/A & Alto & Medio \\
\bottomrule
\end{tabularx}
\end{table}

\subsection{Documentación del desarrollo realizado}

Material técnico disponible en repositorio:
\begin{itemize}
\item Código fuente STM32 (\texttt{Software STM32/main\}).
\item Esquemáticos y PCB (\texttt{Hardware/placa dimmer\}, \texttt{Hardware/placa shield\}).
\item Diagramas de estado (\texttt{Diagrama de Harel\}).
\item App móvil (\texttt{app celular\}).
\item Memoria técnica y contenido gráfico (\texttt{Memoria técnica\}).
\end{itemize}

\hrulefill

\section{Conclusiones}

\subsection{Resultados obtenidos}

Se obtuvo un prototipo funcional que integra:
\begin{itemize}
\item Control local de luz y ventilador.
\item Sincronización con cruce por cero para disparo de TRIAC.
\item Telemetría por Bluetooth HC-06.
\item Persistencia en flash y manejo de falla segura.
\end{itemize}

También se estableció una base sólida de documentación técnica para cierre de entrega final.

El proyecto permitió conocer los Triacs como componentes de control de potencia, además de permitir ahondar en lo que es el desarrollo de sistemas embebidos a pequeña escala.

\subsection{Lecciones aprendidas}

\begin{itemize}
\item El circuito de ZCD actual funciona, pero resulta más complejo de lo necesario para una próxima iteración.
\item La compensación temporal del cruce por cero (aprox. 500 us) es crítica para estabilidad del dimming.
\item La fabricación de PCB artesanal aceleró iteraciones, pero exige mayor cuidado mecánico en footprints de componentes de potencia.
\item La telemetría binaria de 2 bytes simplificó integración y depuración con app móvil.
\end{itemize}

\subsection{Próximos pasos}

\begin{itemize}
\item Evaluar una revisión de hardware con ZCD simplificado, mejor mecánica de placa para componentes de potencia y posible partición de control de dimming en microcontrolador dedicado.
\end{itemize}

\hrulefill

\section{Uso de herramientas de IA}

Se documenta el uso de IA según requerimiento docente y archivo \texttt{listado de cosas hechas con IA.txt\}.

\subsection{Uso individual y conjunto}

\begin{itemize}
\item Ignacio:
\item asistencia para extraer estructura de memoria técnica.
\item apoyo en revisión de README y documentación.
\item apoyo en criterios de hardware y selección de componentes.
\end{itemize}

\begin{itemize}
\item Francisco:
\item soporte para flujo de Itemis Create y diagramas de estado.
\item generación de estructura inicial de documentación técnica de statechart (luego revisada manualmente).
\end{itemize}

\begin{itemize}
\item Uso común del equipo:
\item apoyo en redacción y ajuste de memoria técnica.
\item apoyo extensivo en programación STM32 (estructura, módulos y ajustes).
\item apoyo para redacción de descripciones de PR.
\end{itemize}


\hrulefill

\section{Bibliografía y referencias}

\begin{enumerate}
\item STMicroelectronics, \textit{UM1724 - STM32 Nucleo-64 boards user manual\}.
\item STMicroelectronics, \textit{MB1136 - Electrical Schematic - STM32 Nucleo-64 boards\}.
\item STMicroelectronics, \textit{STM32F103RB Datasheet\}.
\item ON Semiconductor, \textit{MOC3023M Datasheet\}.
\item STMicroelectronics, \textit{BTA06-600C Datasheet / notas de aplicación TRIAC\}.
\item Repositorio del proyecto: \texttt{https://github.com/Embebidos-Fran-Marcos-Nacho/tdse-tf\textbackslash{}\_1-2\}.
\end{enumerate}

Referencias internas del repositorio:
\begin{itemize}
\item \texttt{README.md\}
\item \texttt{Informe\textbackslash{}\_de\textbackslash{}\_Avances.md\}
\item \texttt{Seguimiento.md\}
\item \texttt{Diagrama de Harel/STATECHART\textbackslash{}\_EXPLANATION.md\}
\item \texttt{Memoria técnica/cosas e imagenes para memoria técnica - hardware/*\}
\item \texttt{listado de cosas hechas con IA.txt\}
\end{itemize}

\hrulefill

\textbf{Fin de la Memoria Técnica\}
Autores: Ignacio Ezequiel Cavicchioli, Francisco Javier Moya
Fecha de edición: 18 de febrero de 2026
\end{document}
